\documentclass[12pt,a4paper]{article}
\usepackage[utf8]{inputenc}
\usepackage[T1]{fontenc}
\usepackage{lmodern}
\usepackage{graphicx}
\usepackage{hyperref}
\usepackage{listings}
\usepackage{xcolor}
\usepackage{enumitem}
\usepackage{fancyhdr}
\usepackage{lastpage}

\definecolor{codegreen}{rgb}{0,0.6,0}
\definecolor{codegray}{rgb}{0.5,0.5,0.5}
\definecolor{codepurple}{rgb}{0.58,0,0.82}
\definecolor{backcolour}{rgb}{0.95,0.95,0.92}
\definecolor{darkerlightblue}{rgb}{0.1, 0.3, 0.5}

\lstdefinestyle{mystyle}{
	backgroundcolor=\color{backcolour},   
	commentstyle=\color{codegreen},
	keywordstyle=\color{darkerlightblue},
	numberstyle=\tiny\color{codegray},
	stringstyle=\color{codepurple},
	basicstyle=\ttfamily\footnotesize,
	breakatwhitespace=false,         
	breaklines=true,                 
	captionpos=b,                    
	keepspaces=true,                 
	numbers=left,                    
	numbersep=5pt,                  
	showspaces=false,                
	showstringspaces=false,
	showtabs=false,                  
	tabsize=2
}

\lstset{style=mystyle}

\pagestyle{fancy}
\fancyhf{}
\renewcommand{\headrulewidth}{0pt}
\rfoot{\thepage\ af \pageref{LastPage}}

\title{Forløbsplan}
\author{Anders S. Østergaard}
\date{\today}


\begin{document}
	\pagenumbering{arabic}
	\maketitle
	\thispagestyle{empty}
	
\section*{Undervisningsforløb i IoT (12 dage)}
\begin{enumerate}[leftmargin=*, label=\textbf{Dag \arabic* (3 timer)}]
	
	% Dag 1
	\item Introduktion til Iot, C++ og Grundlæggende Programmering
	\begin{itemize}
		\item \textbf{Teori:}
		\begin{itemize}
			\item Introduktion til IoT C++, syntaks, Output og Comments 
			\item Variables, User Input, Data Types
			\item Operators, Strings og Conditions (if, else-if, else)
			\item Loops (for, while, do-while)
			\item Functions
		\end{itemize}
		\item \textbf{Homework:}
		\begin{itemize}
			\item Laver opgaverne fra Kap. 4 
		\end{itemize}
		\item \textbf{Litteratur:}
		\begin{itemize}
			\item Kap. 1
			\item Kap. 2
		\end{itemize}
	\end{itemize}
	
	% Dag 2
	\item Node-RED
	\begin{itemize}
		\item \textbf{Teori:}
		\begin{itemize}
			\item Introduktion til Node-RED og flow-baseret programmering
			\item Opsætning og grundlæggende brug af Node-RED
		\end{itemize}
		\item \textbf{Home work:}
		\begin{itemize}
			\item Lav opgaverne fra Kap. 7
		\end{itemize}
		\item \textbf{Litteratur:}
		\begin{itemize}
			\item Kap. 8 og 9
		\end{itemize}
	\end{itemize}
	
	% Dag 3
	\item Opgaver
	\begin{itemize}
		\item \textbf{Teori:}
		\begin{itemize}
			\item Ingen teori kun opgaver
		\end{itemize}
		\item  \textbf{Homework:}
		\begin{itemize}
			\item Opgaver fra tidligere som ikke er færdige
		\end{itemize}
		\item \textbf{Litteratur:}
		\begin{itemize}
			\item Tidligere materiale som har behov for at blive læst igen
		\end{itemize}
	\end{itemize}	
	
	% Dag 4
	\item Netværkskommunikation i IoT
	\begin{itemize}
		\item \textbf{Teori:}
		\begin{itemize}
			\item TCP/IP-protokolstakken
			\item IP-adressering og portnumre
			\item Client-server kommunikation
			\item HTTP-protokollen
		\end{itemize}
		\item  \textbf{Homework:}
		\begin{itemize}
			\item Lav opgaverne fra Kap. 3
		\end{itemize}
		\item \textbf{Litteratur:}
		\begin{itemize}
			\item Netværk basic
		\end{itemize}
	\end{itemize}
	
	% Dag 5
	\item Modbus \& HTTP
	\begin{itemize}
		\item \textbf{Teori:}
		\begin{itemize}
			\item Introduktion til Modbus-protokollen
			\item Opgave: Implementering af Modbus-, MQTT- og Coap-kommunikation mellem ESP32 og Node-red
		\end{itemize}
		\item  \textbf{Homework:}
		\begin{itemize}
			\item Lav opgaven fra afs. 15.3 (Modbus)
			\item Lav opgaven fra afs. 15.5 (HTTP)
		\end{itemize}
		\item \textbf{Litteratur:}
		\begin{itemize}
			\item Kap. 10
			\item Kap. 12
		\end{itemize}
	\end{itemize}

	% Dag 6
	\item MQTT \& CoAP
	\begin{itemize}
		\item \textbf{Teori:}
		\begin{itemize}
		    \item MQTT-protokollen og MQTT-brokere
			\item Introduktion til CoAP-protokollen
			\item Opgaver
		\end{itemize}
		\item  \textbf{Homework:}
		\begin{itemize}
			\item Lav opgaven fra afs. 15.4 (MQTT)
			\item Lav opgaven fra afs. 15.7 (CoAP)
		\end{itemize}
		\item \textbf{Litteratur:}
		\begin{itemize}
			\item Kap. 11
			\item Kap. 14
		\end{itemize}
	\end{itemize}
	
	% Dag 7
	\item C++ Advanced
	\begin{itemize}
		\item \textbf{Teori:}
		\begin{itemize}
			\item Classes/Objects
			\item Class Methods
			\item Constructors
			\item Access Specifiers
			\item Encapsulation
			\item Inheritance
			\item  Polymorphism
		\end{itemize}
		\item \textbf{Homework:}
		\begin{itemize}
			\item Laver opgaverne fra Kap. 5
		\end{itemize}
		\item \textbf{Litteratur:}
		\begin{itemize}
			\item Kap. 4
		\end{itemize}
	\end{itemize}

	% Dag 8
	\item Firebase Realtime Database
	\begin{itemize}
		\item \textbf{Teori:}
		\begin{itemize}
			\item Introduktion til Firebase og Realtime Database
			\item Opsætning og konfiguration af Firebase Realtime Database
			\item Opgave: Konfigurer og opsæt forbindelse mellem ESP32- og Node-red til Firebase 
		\end{itemize}
		\item  \textbf{Homework:}
		\begin{itemize}
			\item Ingen med mindre tidligere opgaver ikke er løst
		\end{itemize}
		\item \textbf{Litteratur:}
		\begin{itemize}
			\item API
		\end{itemize}
	\end{itemize}

	% Dag 9
	\item Introduktion til AI
	\begin{itemize}
		\item \textbf{Teori:}
		\begin{itemize}
			\item Kunstig Intelligens (AI) i IoT
			\item Udvikle simpel model
		\end{itemize}
		\item  \textbf{Homework:}
		\begin{itemize}
			\item Ingen med mindre tidligere opgaver ikke er løst
		\end{itemize}
		\item \textbf{Litteratur:}
		\begin{itemize}
			\item AI
		\end{itemize}
	\end{itemize}

	% Dag 10
	\item Projektarbejde og sikkerhed (Del 1)
	\begin{itemize}
		\item Arbejde på projekt
	\end{itemize}
	
	% Dag 11
	\item Projektarbejde og sikkerhed (Del 2)
	\begin{itemize}
		\item Arbejde på projekt
	\end{itemize}

	% Dag 12
	\item Eksamensforberedelse
	\begin{itemize}
		\item Opsamling på emnerne
		\item Gennemgang af øvelser og eksamensrelaterede spørgsmål
	\end{itemize}	
\end{enumerate}
\end{document}
