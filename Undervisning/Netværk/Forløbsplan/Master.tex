\documentclass[12pt,a4paper]{article}
\usepackage[utf8]{inputenc}
\usepackage[T1]{fontenc}
\usepackage{lmodern}
\usepackage{graphicx}
\usepackage{hyperref}
\usepackage{listings}
\usepackage{xcolor}
\usepackage{enumitem}
\usepackage{fancyhdr}
\usepackage{lastpage}

\definecolor{codegreen}{rgb}{0,0.6,0}
\definecolor{codegray}{rgb}{0.5,0.5,0.5}
\definecolor{codepurple}{rgb}{0.58,0,0.82}
\definecolor{backcolour}{rgb}{0.95,0.95,0.92}
\definecolor{darkerlightblue}{rgb}{0.1, 0.3, 0.5}

\lstdefinestyle{mystyle}{
	backgroundcolor=\color{backcolour},   
	commentstyle=\color{codegreen},
	keywordstyle=\color{darkerlightblue},
	numberstyle=\tiny\color{codegray},
	stringstyle=\color{codepurple},
	basicstyle=\ttfamily\footnotesize,
	breakatwhitespace=false,         
	breaklines=true,                 
	captionpos=b,                    
	keepspaces=true,                 
	numbers=left,                    
	numbersep=5pt,                  
	showspaces=false,                
	showstringspaces=false,
	showtabs=false,                  
	tabsize=2
}

\lstset{style=mystyle}

\pagestyle{fancy}
\fancyhf{}
\renewcommand{\headrulewidth}{0pt}
\rfoot{\thepage\ af \pageref{LastPage}}

\title{Forløbsplan}
\author{Anders S. Østergaard}
\date{\today}

\begin{document}
	\pagenumbering{arabic}
	\maketitle
	\thispagestyle{empty}
	\vspace{14cm}
	\noindent\textbf{OBS!} Vær opmærksom på at der kan forekomme ændringer i forløbsplanen
	\clearpage
	\section*{Undervisningsforløb i Industrielt Netværk \\\vspace{-1cm}\begin{center}
			(12 dage)
	\end{center}}
	
	\begin{enumerate}[leftmargin=*, label=\textbf{Dag \arabic* (3 timer)}]
		
		% Dag 1
		\item IT-Netværk Del 1
		\begin{itemize}
			\item \textbf{Teori (1,5 time):}
			\begin{itemize}
				\item Grundlæggende Netværksbegreber 
				\item Netværkshardware 
				\item Forbindelsesorienteret vs. Forbindelsesløse Netværk
				\item Media Access Control (MAC) mechanisms
				\item Transmissionsteknikker
				\item OSI-modellen
				\item TCP/IP-modellen
			\end{itemize}
			\item \textbf{Praktisk arbejde (1,5 time):}
			\begin{itemize}
				\item \textbf{Cisco Packet Tracer}
				\begin{itemize}
					\item Hub
					\item Switch
				\end{itemize}
			\end{itemize}
			\item \textbf{Litteratur}
			\begin{itemize}
				\item Læs kap.: 1 og afs. 2.1 til 2.16
			\end{itemize}
			\item \textbf{Hjemmeopgave:}
			\begin{itemize}
				\item Opgave [1]: Tegn og beskriv et netværksdiagram for et lille kontornetværk med 8 computere, i et LAN.
				
				\item Opgave [2]: Lav opgaver Hub og Switch under afsnittet \textbf{Cisco Packet Tracer} færdigt
				
				\item Opgave [3]: Skriv kort refleksioner omkring Hub, Switch, hvad er en IP-adresse og subnetting.
			\end{itemize}
		\end{itemize}
		
		% Dag 2
		\item IT-Netværk Del 2
		\begin{itemize}
			\item \textbf{Teori (1,5 time):}
			\begin{itemize}
				\item IPv4 
				\item Subnetting
				\item Classless Inter-Domain Routing (CIDR)
				\item DHCP, DNS, NAT, og VLAN 
				\item Kabling
			\end{itemize}
			\item \textbf{Praktisk arbejde (1,5 time):}
			\begin{itemize}
				\item Router
				\item VPN
				\item VLAN
				\item NAT
				\item DHCP
				\item DNS
			\end{itemize}
			\item \textbf{Litteratur}
			\begin{itemize}
				\item Læs Kap. 3
			\end{itemize}
			
			\item \textbf{Hjemmeopgave:}
			\begin{itemize}
				\item Opgave [1]: Lav opgaverne under afsnittet \\\textbf{Cisco Packet Tracer}
				\begin{itemize}
					\item Router
					\item VLAN
					\item VPN
					\item NAT
					\item DHCP
					\item DNS
				\end{itemize}
			\end{itemize}
		\end{itemize}
		\clearpage
		
		% Dag 3
		\item Industrielle Netværksprotokoller og Teknologier - Dag I
		\begin{itemize}
			\item \textbf{Teori (1,5 time):}
			\begin{itemize}
				\item Industrielle Netværkstopologier
				\item Grundlæggende Serielle \& parallelle kommunikation 
				\item Fieldbus protokoller
			\end{itemize}
			\item \textbf{Praktisk arbejde (1,5 time):}
			\begin{itemize}
				\item Profibus (aflevering).
				\item Modbus (Simens/Rockwell) og UR simulator (aflevering).
			\end{itemize}
			\item \textbf{Litteratur}
			\begin{itemize}
				\item Læs kap. 7, 8 og afs. 9.1 og 9.3
			\end{itemize}
			\item \textbf{Hjemmeopgave:}
			\begin{itemize}
				\item Færdiggørelse af øvelse
			\end{itemize}
		\end{itemize}
		
		% Dag 4
		\item Industrielle Netværksprotokoller og Teknologier - Dag II
		\begin{itemize}
			\item \textbf{Teori (1 time):}
			\begin{itemize}
				\item Profinet
			\end{itemize}
			\item \textbf{Praktisk arbejde (2 time):}
			\begin{itemize}
				\item Siemens 
			\end{itemize}
			\item \textbf{Litteratur}
			\begin{itemize}
				\item Læs Afs. 9.3
			\end{itemize}
			\item \textbf{Hjemmeopgave:}
			\begin{itemize}
				\item Opgave [1]: Lav opgaverne færdig under Kap. 12
			\end{itemize}
		\end{itemize}
		
		% Dag 5
		\item Industrielle Netværksprotokoller og Teknologier - Dag III
		\begin{itemize}
			\item \textbf{Teori (1,5 time):}
			\begin{itemize}
				\item AS-i
				\item IO-Link
				\item Ethernet/IP.
			\end{itemize}
			\item \textbf{Praktisk arbejde (1,5 time):}
			\begin{itemize}
				\item Rockwell + Rockwell aflevering
			\end{itemize}
			\item \textbf{Litteratur}
			\begin{itemize}
				\item Læs Afs. 9.2, 9.4, 9.5
			\end{itemize}
			\item \textbf{Hjemmeopgave:}
			\begin{itemize}
				\item Opgave [1]: Lav opgaverne færdige som omhandler Siemens og Rockwell + Rockwell aflevering
			\end{itemize}
		\end{itemize}

		% Dag 6
		\item Industrielle Netværksprotokoller og Teknologier - Dag IIII
		\begin{itemize}
			\item \textbf{Teori (1,5 time):}
			\begin{itemize}
				\item KepServerEX
				\item OPC
			\end{itemize}
			\item \textbf{Praktisk arbejde (1,5 time):}
			\begin{itemize}
				\item KepServerEX opgaver + aflevering
			\end{itemize}
			\item \textbf{Litteratur}
			\begin{itemize}
				\item Læs Afs. 9.6 og 9.7
			\end{itemize}
			\item \textbf{Hjemmeopgave:}
			\begin{itemize}
				\item Opgave [1]: Løs opgaverne vedr. KepServerEX og OPC + aflevering
			\end{itemize}
		\end{itemize}
		
		% Dag 7
		\item Robot kommunikation
		\begin{itemize}
			\item \textbf{Praktisk arbejde (3 time):}
			\begin{itemize}
				\item UR, AUBO, Rockwell, ESP32 (optional)
			\end{itemize}
			\item \textbf{Hjemmeopgave:}
			\begin{itemize}
				\item Opgave [1]: UR 
				\item Opgave [2]: AUBO
			\end{itemize}
		\end{itemize}
		
		% Dag 8
		\item Netværkssikkerhed og Problemløsning 
		\begin{itemize}
			\item \textbf{Teori (1,5 time):}
			\begin{itemize}
				\item Netværkssikkerhedsprincipper 
				\item Firewalls og VPN-konfiguration 
			\end{itemize}
			\item \textbf{Praktisk arbejde (1,5 time):}
			\begin{itemize}
				\item Implementering af en firewall
			\end{itemize}
			\item \textbf{Hjemmeopgave:}
			\begin{itemize}
				\item Gå sammen 3 i en gruppe og lave en præsentation om en netværksteknologi og præsenter den i klassen til næste lektion. Præsentationen skal være mellem 8-9 min lang.
			\end{itemize}
		\end{itemize}

		% Dag 9
		\item Præsentation af netværksteknologier af studerende
		\begin{itemize}
			\item \textbf{Teori:}
			\begin{itemize}
				\item Gennemgang Præsentation fra studerende
			\end{itemize}
		\end{itemize}
		
		% Dag 10
		\item Gennemgang af opgaver
		\begin{itemize}
			\item \textbf{Teori og praktisk opsummering (3 timer):}
			\begin{itemize}
				\item Gennemgang af centrale emner, opgaver og opsamling på uafklarede spørgsmål.
			\end{itemize}
		\end{itemize}

		% Dag 11
		\item Forberedelse til Tværfagligt Projekt (Dag 11)
		\begin{itemize}
			\item \textbf{Teori og praktisk opsummering (3 timer):}
			\begin{itemize}
				\item Sidste gennemgang af centrale emner og deres anvendelse i det tværfaglige projekt.
				\item \textbf{Ekstra Opgaver!} (For dem der har gennemført alle de øvrige opgaver):
				\begin{itemize}
					\item ESP32/NodeMCU 8266: Implementering og test af netværkskommunikation med UR robotter, Siemens, Rockwell.
					\item Node-Red: Implementering af kommunikation og udtræk af data fra PLC'er
					\item Python: Implementering af kommunikation og udtræk af data fra PLC'er
				\end{itemize}
			\end{itemize}
		\end{itemize}
		
			% Dag 12
		\item Forberedelse til Tværfagligt Projekt (Dag 12)
		\begin{itemize}
			\item \textbf{Teori og praktisk opsummering (3 timer):}
			\begin{itemize}
				\item Sidste gennemgang af centrale emner og deres anvendelse i det tværfaglige projekt.				
			\end{itemize}
		\end{itemize}
	\end{enumerate}
\end{document}

